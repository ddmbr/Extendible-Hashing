\documentclass{article}
\usepackage{xeCJK}
\linespread{1.1}
\setCJKmainfont{WenQuanYi Micro Hei}
\title{Extendible Hashing项目实验报告}
\author{10389048杨帆 \\ 10389084梁展瑞 \\ 12xxxxxx古宣佑}
\begin{document}
\maketitle
\section{摘要}
\section{实验环境与工具}
    \begin{enumerate}
        \item Operating system: Linux(Debian \& Arch \& Ubuntu, with 3.0+ kernels)
        \item Language: C
        \item Compiler: GCC
        \item Documentation: \LaTeX , asciidoc, Markdown
        \item Collaboration Platform/Project Hosting: GitHub
        \item Debug utils: GDB, python, perf, gprof, xdot, gprof2dot, VIM(to do regex matching often)
        \item Editor: VIM
    \end{enumerate}
    \paragraph{}
        这是在GitHub上的repo,可以看到多个branch及历史记录。
    \begin{verbatim}
https://github.com/ddmbr/Extendible-Hashing/
    \end{verbatim}
\section{流程总览}
    \paragraph{}
        整个流程可以划分为建立和查询两个大方面,分别叙述如下。(repo有初期使用的草稿:draft.txt)
    \subsection{建立数据库}
        \paragraph{}
            考虑到一个情况,当同时有大量的相同key的数据插入时,会导致一个桶无法分裂而必须拉链。但事前的统计发现,
        \begin{enumerate}
            \item key相同的条目最多只有7条。
            \item 依照我们的实现方式,每个桶的容量大概在100条。
        \end{enumerate}
        \paragraph{}
            所以我们的程序没有加入拉链的方法。以下是简要流程。
\begin{verbatim}
While the end of the raw file(i.e, lineitem.tbl) is not reached, 
    Load a page from the raw file to the memory.
    Loop through the page, to:
        Read the next record in the page.
        Get its key.
        Get the hash value `hv' of the key.
        Fetch the corresponding index page,
        According to the index, fetch the corresponding
        bucket page.
        Before inserting the record, check that 
        whether the bucket will be overflowed.
            If yes,
                If global depth == local depth,
                    Double the index
            Then split the bucket and redistribute.
        Write the record into the page.

\end{verbatim}
    \subsection{查询}
\begin{verbatim}
While not reach the end of the query,
    Read next specific key.
    Get the hash value `hv' of the key.
    Fetch the corresponding bucket
    Loop through the bucket and print the matched records.
\end{verbatim}
\section{架构}
    \subsection{总览}
        \paragraph{}
            程序大致划分为4个模块,分别为File Manager, Buffer Manager, Parser以及Hash。
        \begin{itemize}
            \item File Manager主要跟磁盘相关的操作,例如在磁盘上申请新页。
            \item Buffer Manager处理跟内存管理相关的操作,例如换页等,时钟页面算法的核心在这个模块里。
            \item Parser处理和源文本文件(lineitem.tbl)读取相关的操作。
            \item Hash模块则处理和哈希算法相关的操作,含有Extendible Hashing的核心。
        \end{itemize}
        \paragraph{}
            在数据结构上,设计了page\_t和record\_t两个重要的结构体,用于存储页(使用void*)和存储记录,附带一些必要信息,如页号,并相应有多种相关方法。比如,将record\_t转换成可以输出的字符串,在page\_t中的页面二进制数据中抓取出指定的记录并转成record\_t,等等。完整方法详见src/ehdb\_record.h,已附有详细注释。
    \subsection{关键实现说明}
        \begin{enumerate}
            \item 时钟页面算法
                \paragraph{}
                    TODO
            \item 哈希算法
                \paragraph{}
                    TODO
        \end{enumerate}
\section{结果与分析}
\section{总结}
\end{document}
